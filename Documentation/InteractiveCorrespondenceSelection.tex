\documentclass{InsightArticle}

\usepackage[dvips]{graphicx}
\usepackage{float}
\usepackage{subfigure}

\usepackage[dvips,
bookmarks,
bookmarksopen,
backref,
colorlinks,linkcolor={blue},citecolor={blue},urlcolor={blue},
]{hyperref}

\title{Interactive Image Correspondence Selection}

% 
% NOTE: This is the last number of the "handle" URL that 
% The Insight Journal assigns to your paper as part of the
% submission process. Please replace the number "1338" with
% the actual handle number that you get assigned.
%
\newcommand{\IJhandlerIDnumber}{3233}

% Increment the release number whenever significant changes are made.
% The author and/or editor can define 'significant' however they like.
\release{0.00}

% At minimum, give your name and an email address.  You can include a
% snail-mail address if you like.

\author{David Doria}
\authoraddress{Rensselaer Polytechnic Institute, Troy NY}


\begin{document}

\IJhandlefooter{\IJhandlerIDnumber}


\ifpdf
\else
   %
   % Commands for including Graphics when using latex
   % 
   \DeclareGraphicsExtensions{.eps,.jpg,.gif,.tiff,.bmp,.png}
   \DeclareGraphicsRule{.jpg}{eps}{.jpg.bb}{`convert #1 eps:-}
   \DeclareGraphicsRule{.gif}{eps}{.gif.bb}{`convert #1 eps:-}
   \DeclareGraphicsRule{.tiff}{eps}{.tiff.bb}{`convert #1 eps:-}
   \DeclareGraphicsRule{.bmp}{eps}{.bmp.bb}{`convert #1 eps:-}
   \DeclareGraphicsRule{.png}{eps}{.png.bb}{`convert #1 eps:-}
\fi


\maketitle


\ifhtml
\chapter*{Front Matter\label{front}}
\fi

\begin{abstract}
\noindent
This document presents an application to manually select corresponding points in two images. The functionality is equivalent to Matlab's 'cpselect' function.

\end{abstract}

\IJhandlenote{\IJhandlerIDnumber}

\tableofcontents
\section{Introduction}
This document presents an application to manually select corresponding points in two images. The functionality is equivalent to Matlab's 'cpselect' function.

%%%%%%%%%%%%%%%%
\section{GUI}
\label{sec:GUI}

\begin{center}
	\begin{figure}[H]
  \centering
		\includegraphics[width=0.6\linewidth]{images/Environment}
		\caption{Default Environment}
		\label{fig:Environment}
	\end{figure}
\end{center} 

% 
% \begin{figure}[H]
% \centering
% \subfigure[Polar coordinates of normal vector]{
% \includegraphics[width=0.33\textwidth]{images/polar1}
% \label{fig:polar}
% }
% \subfigure[Hough Transform of $3$ points] {
% \includegraphics[width=0.33\textwidth]{images/hough3}
% \label{fig:sinusoid}
% }
% \subfigure[Accumulator] {
% \includegraphics[width=0.25\textwidth]{images/accumulator}
% \label{fig:accumulator}
% }
% \caption{The Hough Transform}
% \end{figure}

%%%%%%%%%%%%%%%
% \begin{thebibliography}{9}
% 
% 	\bibitem{boykov2006}
% 	  Yuri Boykov,
% 	  \emph{Graph Cuts and Efficient N-D Image Segmentation}.
% 	  International Journal of Computer Vision
% 	  2006
% 
% \end{thebibliography}

\end{document}